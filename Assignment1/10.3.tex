\let\negmedspace\undefined
\let\negthickspace\undefined
\documentclass{article}
\usepackage{cite}
\usepackage{amsmath,amssymb,amsfonts,amsthm}
\usepackage{algorithmic}
\usepackage{graphicx}
\usepackage{textcomp}
\usepackage{xcolor}
\usepackage{txfonts}
\usepackage{listings}
\usepackage{enumitem}
\usepackage{tfrupee}
\usepackage{mathtools}
\usepackage{gensymb}
\usepackage[breaklinks=true]{hyperref}
\usepackage{tkz-euclide} % loads  TikZ and tkz-base
\usepackage{listings}
%\usepackage{gvv}
%
%\usepackage{setspace}
%\usepackage{gensymb}
%\doublespacing
%\singlespacing

%\usepackage{graphicx}
%\usepackage{amssymb}
%\usepackage{relsize}
%\usepackage[cmex10]{amsmath}
%\usepackage{amsthm}
%\interdisplaylinepenalty=2500
%\savesymbol{iint}
%\usepackage{txfonts}
%\restoresymbol{TXF}{iint}
%\usepackage{wasysym}
%\usepackage{amsthm}
%\usepackage{iithtlc}
%\usepackage{mathrsfs}
%\usepackage{txfonts}
%\usepackage{stfloats}
%\usepackage{bm}
%\usepackage{cite}
%\usepackage{cases}
%\usepackage{subfig}
%\usepackage{xtab}
%\usepackage{longtable}
%\usepackage{multirow}
%\usepackage{algorithm}
%\usepackage{algpseudocode}
%\usepackage{enumitem}
%\usepackage{mathtools}
%\usepackage{tikz}
%\usepackage{circuitikz}
%\usepackage{verbatim}
%\usepackage{tfrupee}
%\usepackage{stmaryrd}
%\usetkzobj{all}
%    \usepackage{color}                                            %%
%    \usepackage{array}                                            %%
%    \usepackage{longtable}                                        %%
%    \usepackage{calc}                                             %%
%    \usepackage{multirow}                                         %%
%    \usepackage{hhline}                                           %%
%    \usepackage{ifthen}                                           %%
  %optionally (for landscape tables embedded in another document): %%
%    \usepackage{lscape}     
%\usepackage{multicol}
%\usepackage{chngcntr}
%\usepackage{enumerate}

%\usepackage{wasysym}
%\documentclass[conference]{IEEEtran}
%\IEEEoverridecommandlockouts
% The preceding line is only needed to identify funding in the first footnote. If that is unneeded, please comment it out.

\newtheorem{theorem}{Theorem}[section]
\newtheorem{problem}{Problem}
\newtheorem{proposition}{Proposition}[section]
\newtheorem{lemma}{Lemma}[section]
\newtheorem{corollary}[theorem]{Corollary}
\newtheorem{example}{Example}[section]
\newtheorem{definition}[problem]{Definition}
%\newtheorem{thm}{Theorem}[section] 
%\newtheorem{defn}[thm]{Definition}
%\newtheorem{algorithm}{Algorithm}[section]
%\newtheorem{cor}{Corollary}
\newcommand{\BEQA}{\begin{eqnarray}}
\newcommand{\EEQA}{\end{eqnarray}}
\newcommand{\define}{\stackrel{\triangle}{=}}
\theoremstyle{remark}
\newtheorem{rem}{Remark}

%\bibliographystyle{ieeetr}
\begin{document}
\title{Latex Assignment1}
\author{APARNA ANAND}
\date{25 August,2023}
\maketitle
\section*{Example 10.3}
\begin{enumerate}
\item Let us take the example given in Section 3.1. Akhila goes to a fair with \rupee~20 and wants to have rides on the Giant Wheel and play Hoopla. Represent this situation algebraically and graphically (geometrically).
\item Romila went to a stationery shop and purchased $2$ pencils and $3$ erasers for \rupee~9. Her friend Sonali saw the new variety of pencils and erasers with Romila, and she also bought $4$ pencils and $6$ erasers of the same kind for \rupee~18. Represent this situation algebraically and graphically.
\item Two rails are represented by the equations $x+2y-4=0$ and $2x+4y-12=0$. Represent this situation geometrically.
\item Check graphically whether the pair of equations.
\begin{align}
x+3y = 6 \\ \text{and} 2x-3y = 12
\end{align}
is consistent. If so, Solve them graphically.
\item Graphically, find whether the following pair of equatons has no solution, unique solution or infinitely many solutions.
\begin{align}
5x-8y+1 = 0 \\ 3x-\frac{24}{5}y+\frac{3}{5} = 0
\end{align}
\item Champa went to a \textquotedblleft Sale \textquotedblright to purchase some pants and skirts. When her friends asked her how many of each she had boughtc she answered, \textquotedblleft The number of skirts is two less than twice the number of pants purchased. Also, the number of skirts is four less than four times the number of pants purchased\textquotedblright. Help her friends to find how many pants and skirts Champa bought.
\item Solve the following pair of equations by substitution method:
\begin{align}
7x-15y = 2 \\  x+2y = 3
\end{align}
\item Solve Q.1 of Exercise 3.1 by the method of substitution.
\\ Aftab tells his daughter, \textquotedblleft Seven years ago, I was seven times as old as you were then. Also, three years from now, I shall be three times as old as you will be.\textquotedblright  (Isn't this interesting ?) Represent this situation algebraically and graphically.
\item Let us consider Example 2 in Section $3.3$ i.e., the cost of 2 pencils and 3 erasers is \rupee~9 and the cost of 4 pencils and 6 erasers is \rupee~18. Find the cost of each pencil and each eraser.
\item Let us consider the Example $3$ of Section $3.2$. Will the rails cross each other? Two rails are represented by the equations $x+2y-4=0$ and $2x+4y-12=0$. Represent this situation geometrically.
\item The ratio of incomes of two persons is 9:7 and the ratio of their expenditures is $4:3$. If each of them manages to save \rupee~2000 per month, find their monthly incomes.
\item Use elimination method to find all possible solutions of the following pair of linear equations:-
\begin{align}
2x+3y = 8 \\ 4x+6y = 7
\end{align}
\item The sum of a two digit number and the number obtained by reversing the digits is $66$.If the digits of the number differ by 2, find the number. How many such numberes are there?
\item From a bus stand in Bangalore, if we buy $2$ tickets to Malleshwaram and $2$ tickets to yeshwanthpur the total cost is \rupee~74. Find the fare from the bus stand to Malleshwaram, and to Yeshwanthpur.
\item For which values $p$ does the pair of equations given below has unique solution.
\begin{align}
4x+py+8 = 0 \\ 2x+2y+2 = 0
\end{align}
\item For what values of $k$ will the following pair of linear equations have infinitely many solutions.
\begin{align}
kx+3y-(k-3) = 0 \\ 12x+ky-k = 0
\end{align}
\item Solve the pair of equations:
\begin{align}
\frac{2}{x}+\frac{3}{y}= 13 \\ \frac{5}{x}+\frac{4}{y} = -2
\end{align}
\item Solve the following pair of linear equations by reducing them to a pair of linear equations
\begin{align}
\frac{5}{x-1}+\frac{1}{y-2}= 2 \\ \frac{6}{x-1}-\frac{3}{y-2}= 1
\end{align}
\item A boat goes $30$ km upstream and $44$ km downstream in $10$ hours. In $13$ hours, it can go $40$ km upstream and $55$ km down-stream. Determine the speed of the stream and that of the boat in still water.
\end{enumerate}
\end{document}


  
 
