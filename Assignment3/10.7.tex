\let\negmedspace\undefined
\let\negthickspace\undefined
\documentclass{article}
\usepackage{cite}
\usepackage{amsmath,amssymb,amsfonts,amsthm}
\usepackage{algorithmic}
\usepackage{graphicx}
\usepackage{textcomp}
\usepackage{xcolor}
\usepackage{txfonts}
\usepackage{listings}
\usepackage{enumitem}
\usepackage{tfrupee}
\usepackage{mathtools}
\usepackage{gensymb}
\usepackage[breaklinks=true]{hyperref}
\usepackage{tkz-euclide} % loads  TikZ and tkz-base
\usepackage{listings}
\usepackage{gvv}
%
%\usepackage{setspace}
%\usepackage{gensymb}
%\doublespacing
%\singlespacing

%\usepackage{graphicx}
%\usepackage{amssymb}
%\usepackage{relsize}
%\usepackage[cmex10]{amsmath}
%\usepackage{amsthm}
%\interdisplaylinepenalty=2500
%\savesymbol{iint}
%\usepackage{txfonts}
%\restoresymbol{TXF}{iint}
%\usepackage{wasysym}
%\usepackage{amsthm}
%\usepackage{iithtlc}
%\usepackage{mathrsfs}
%\usepackage{txfonts}
%\usepackage{stfloats}
%\usepackage{bm}
%\usepackage{cite}
%\usepackage{cases}
%\usepackage{subfig}
%\usepackage{xtab}
%\usepackage{longtable}
%\usepackage{multirow}
%\usepackage{algorithm}
%\usepackage{algpseudocode}
%\usepackage{enumitem}
%\usepackage{mathtools}
%\usepackage{tikz}
%\usepackage{circuitikz}
%\usepackage{verbatim}
%\usepackage{tfrupee}
%\usepackage{stmaryrd}
%\usetkzobj{all}
%    \usepackage{color}                                            %%
%    \usepackage{array}                                            %%
%    \usepackage{longtable}                                        %%
%    \usepackage{calc}                                             %%
%    \usepackage{multirow}                                         %%
%    \usepackage{hhline}                                           %%
%    \usepackage{ifthen}                                           %%
  %optionally (for landscape tables embedded in another document): %%
%    \usepackage{lscape}
%\usepackage{multicol}
%\usepackage{chngcntr}
%\usepackage{enumerate}

%\usepackage{wasysym}
%\documentclass[conference]{IEEEtran}
%\IEEEoverridecommandlockouts
% The preceding line is only needed to identify funding in the first footnote. If that is unneeded, please comment it out.

\newtheorem{theorem}{Theorem}[section]
\newtheorem{problem}{Problem}
\newtheorem{proposition}{Proposition}[section]
\newtheorem{lemma}{Lemma}[section]
\newtheorem{corollary}[theorem]{Corollary}
\newtheorem{example}{Example}[section]
\newtheorem{definition}[problem]{Definition}
%\newtheorem{thm}{Theorem}[section]
%\newtheorem{defn}[thm]{Definition}
%\newtheorem{algorithm}{Algorithm}[section]
%\newtheorem{cor}{Corollary}
\newcommand{\BEQA}{\begin{eqnarray}}
\newcommand{\EEQA}{\end{eqnarray}}
%\newcommand{\define}{\stackrel{\triangle}{=}}
\theoremstyle{remark}
\newtheorem{rem}{Remark}

%\bibliographystyle{ieeetr}
\begin{document}
\title{Latex Assignment1}
\author{APARNA ANAND}
\date{29 August,2023}
\maketitle
\section*{Example-1-15 (10.7)}
\begin{enumerate}
\item Do the points $(3,2), (-2,-3)$ and $(2,3)$ form a triangle? If so, name the type of triangle formed.
\item Show that the points $(1,7),(4,2),(-1,-1)$ and $(-4,4)$ are the vertices of a square.
\item \figref{fig:7.6} shows the arrangement of desks in a classroom. Ashima, Bharti and Camella are seated at $A(3,1),B(6,4)$ and $C(8,6)$ respectively. Do you think they are seated in a line? Give reasons for your answer.
\begin{figure}[h]
\centering 
\includegraphics[width=\columnwidth]{figs/7.6.png}
\caption{7.6}
\label{fig:7.6}
\end{figure}
\item Find a relation between $x$ and $y$ such that the point $(x,y)$ is equidistant from the points $(7,1)$ and $(3,5)$.
\item Find a point on the Y-axis which is equidistant from the points $A(6,5)$ and $B(-4,3)$.
\item Find the coordinates of the point which divides the line segment joining the points $(4,-3)$ and $(8,5)$ in the ratio $3:1$ internally.
\item In what ratio does the point $(-4,6)$ divide the line segment joining the points $A(-6,0)$ and $B(3,-8)$?
\item Find the coordinates of the points of trisection (i.e. points dividing to three equal parts) of the line segment joining the points $A(2,-2)$ and $B(-7,4)$.
\item Find the ratio in which the Y-axis divides the line segment joining the points $(5,-6)$ and $(-1,-4)$. Also find the point of intersection.
\item If the points $A(6,1),B(8,2), C(9,4)$ and $D(p,3)$ are the vertices of a parallelogram, taken in order, find the value of $p$.
\item Find the area of the triangle whose vertices are $(1,-1), (-4,6)$ and $(-3,5)$.
\item Find the area of a triangle formed by the points $A(5,2), B(4,7)$ and $(7,-4)$.
\item Find the area of the triangle formed by the points $P(-1.5,3), Q(6,-2)$ and $R(-3,4)$.
\item Find the values of $k$ if the points $A(2,3), B(4,k)$ and $C(6,-3)$ are collinear.
\item If $A(-5,7), B(-4,-5), C(-1,-6)$ and $D(4,5)$ are the vertices of a quadrilateral, find the area of quadrilateral ABCD.
\end{enumerate}
\end{document}
 

