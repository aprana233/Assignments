\let\negmedspace\undefined
\let\negthickspace\undefined
\documentclass{article}
\usepackage{cite}
\usepackage{amsmath,amssymb,amsfonts,amsthm}
\usepackage{algorithmic}
\usepackage{graphicx}
\usepackage{textcomp}
\usepackage{xcolor}
\usepackage{txfonts}
\usepackage{listings}
\usepackage{enumitem}
\usepackage{mathtools}
\usepackage{gensymb}
\usepackage[breaklinks=true]{hyperref}
\usepackage{tkz-euclide} % loads  TikZ and tkz-base
\usepackage{listings}
\usepackage{gvv}
%
%\usepackage{setspace}
%\usepackage{gensymb}
%\doublespacing
%\singlespacing

%\usepackage{graphicx}
%\usepackage{amssymb}
%\usepackage{relsize}
%\usepackage[cmex10]{amsmath}
%\usepackage{amsthm}
%\interdisplaylinepenalty=2500
%\savesymbol{iint}
%\usepackage{txfonts}
%\restoresymbol{TXF}{iint}
%\usepackage{wasysym}
%\usepackage{amsthm}
%\usepackage{iithtlc}
%\usepackage{mathrsfs}
%\usepackage{txfonts}
%\usepackage{stfloats}
%\usepackage{bm}
%\usepackage{cite}
%\usepackage{cases}
%\usepackage{subfig}
%\usepackage{xtab}
%\usepackage{longtable}
%\usepackage{multirow}
%\usepackage{algorithm}
%\usepackage{algpseudocode}
%\usepackage{enumitem}
%\usepackage{mathtools}
%\usepackage{tikz}
%\usepackage{circuitikz}
%\usepackage{verbatim}
%\usepackage{tfrupee}
%\usepackage{stmaryrd}
%\usetkzobj{all}
%    \usepackage{color}                                            %%
%    \usepackage{array}                                            %%
%    \usepackage{longtable}                                        %%
%    \usepackage{calc}                                             %%
%    \usepackage{multirow}                                         %%
%    \usepackage{hhline}                                           %%
%    \usepackage{ifthen}                                           %%
  %optionally (for landscape tables embedded in another document): %%
%    \usepackage{lscape}     
%\usepackage{multicol}
%\usepackage{chngcntr}
%\usepackage{enumerate}

%\usepackage{wasysym}
%\documentclass[conference]{IEEEtran}
%\IEEEoverridecommandlockouts
% The preceding line is only needed to identify funding in the first footnote. If that is unneeded, please comment it out.

\newtheorem{theorem}{Theorem}[section]
\newtheorem{problem}{Problem}
\newtheorem{proposition}{Proposition}[section]
\newtheorem{lemma}{Lemma}[section]
\newtheorem{corollary}[theorem]{Corollary}
\newtheorem{example}{Example}[section]
\newtheorem{definition}[problem]{Definition}
%\newtheorem{thm}{Theorem}[section] 
%\newtheorem{defn}[thm]{Definition}
%\newtheorem{algorithm}{Algorithm}[section]
%\newtheorem{cor}{Corollary}
\newcommand{\BEQA}{\begin{eqnarray}}
\newcommand{\EEQA}{\end{eqnarray}}
%\newcommand{\define}{\stackrel{\triangle}{=}}
\theoremstyle{remark}
\newtheorem{rem}{Remark}

%\bibliographystyle{ieeetr}
\begin{document}
\title{Latex Assignment9}
\author{Aparna}
\maketitle
\section*{Example 1-30 (12.11)}
\begin{enumerate}
\item If a line makes angle $90 \degree, 60 \degree$ and $30 \degree$ with the positive direction of x, y and z-axes respectively, find its direction cosines.
\item If a line has direction ratios $2, -1, -2$, determine its direction cosines.
\item Find the direction cosines of the line passing through the two points $(-2, 4, -5)$ and $(1, 2, 3)$.
\item Find the direction cosines of x, y and z-axis.
\item Show that the points $A(2, 3, -4), B(1, -2, 3)$ and $C(3, 8, -11)$ are collinear.
\item Find the vector and the cartesian equations of the line through the point $(5, 2, -4)$ and which is parallel to the vector $3 \hat{i}+2 \hat{j}- 8 \hat{k}$.
\item Find the vector equation for the line passing through the points $(-1, 0, 2)$ and $(3, 4, 6)$.
\item The Cartesian equation of a line is
\begin{align}
\frac{x+3}{2}= \frac{y-5}{4}= \frac{z+6}{2}
\end{align} 
Find the vector equation for the line.
\item Find the angle between the pair of lines given by
\begin{align}
\overrightarrow{r}= 3 \hat{i}+ 2 \hat{j}- 4 \hat{k}+ \lambda(\hat{i}+ 2 \hat{j}+ 2 \hat{k}) \\
\text{ and } \overrightarrow{r}= 5 \hat{i}+ 2 \hat{j}+ \mu(3 \hat{i}+ 2 \hat{j}+ 6 \hat{k}) 
\end{align}
\item Find the angle between the pair of lines:
\begin{align}
\frac{x+3}{3}= \frac{y-1}{5}= \frac{z+3}{4}\\
\text{ and }\frac{x+1}{1}= \frac{y-4}{1}= \frac{z+5}{2}
\end{align}
\item Find the shorter distance between the lines $l_1$ and $l_2$ whose vector equations are
\begin{align}
\overrightarrow{r}= \hat{i}+ \hat{j}+ \lambda(2 \hat{i}- \hat{j}+ \hat{k}) \\
\text{ and } \overrightarrow{r}= 2 \hat{i}+ \hat{j}- \hat{k}+ \mu(3 \hat{i}- 5 \hat{j}+ 2 \hat{k}) 
\end{align}
\item Find the distance between the lines $l_1$ and $l_2$ given by
\begin{align}
\overrightarrow{r}= \hat{i}+ 2 \hat{j}- 4 \hat{k}+ \lambda(2 \hat{i}+ 3 \hat{j}+ 6 \hat{k}) \\
\text{ and }\overrightarrow{r}= 3 \hat{i}+ 3 \hat{j}- 5 \hat{k}+ \mu(2 \hat{i}+ 3 \hat{j}+ 6 \hat{k}) 
\end{align}
\item Find the vector equation of the plane which is at a distance of $\frac{6}{\sqrt{29}}$ from the origin and its normal vector from the origin $2 \hat{i}- 3 \hat{j}+ 4 \hat{k}$. Also find its cartesian form.
\item Find the direction cosines of the unit vector perpendicular to the plane $\overrightarrow{r} \cdot(6 \hat{i}- 3 \hat{j}- 2 \hat{k})+ 1= 0$ passing through the origin.
\item Find the distance of the plane $2x- 3y+ 4z- 6= 0$ from the origin.
\item Find the coordinates of the foot of the perpendicular drawn from the origin to the plane $2x -3y +4z -6 = 0$.
\item Find the vector and cartesian equations of the plane which passes through the point $(5, 2, -4)$ and perpendicular to the line with direction ratios $2, 3, -1$.
\item Find the vector equations of the plane passing through the points $R(2, 5, -3), S(-2, -3, 5)$ and $T(5, 3, -3)$.
\item Find the equation of the plane with intercepts $2, 3$ and $4$ on the x, y and z - axis respectively.
\item Find the vector equation of the plane passing through the intersection of the planes $\overrightarrow{r} \cdot (\hat{i} +\hat{j} +\hat{k})=6$ and $\overrightarrow{r} \cdot (2\hat{i} +3\hat{j} +4\hat{k})=-5$, and the point $(1, 1, 1)$.
\item Show that the lines 
\begin{align}
\frac{x+3}{-3}= \frac{y-1}{1}= \frac{z-5}{5} \text{ and } \frac{x+1}{-1} =\frac{y-2}{2} =\frac{z-5}{5} \text{ are coplanar }.
\end{align}
\item Find the angle between the two planes $2x +y -2z =5$ and $3x- 6y- 2z= 7$ using vector method.
\item Find the angle between the two planes $3x -6y +2z =7$ and $2x +2y -2z =5$.
\item Find the distance of a point $(2, 5, -3)$ from the plane $\overrightarrow{r} \cdot (6\hat{i} -3\hat{j} +2\hat{k}) =4$.
\item Find the angle between the line $\frac{x+1}{2} =\frac{y}{3} =\frac{z-3}{6}$ and the plane $10x +2y -11z =3$.
\item A line makes angles $\alpha, \beta, \gamma$ and $\delta$  with the diagonals of a cube, prove that 
\begin{align}
\cos^2\alpha +\cos^2\beta +\cos^2\gamma +\cos^2\delta = \frac{4}{3}.
\end{align}
\item Find the equation of the plane that contains the point $(1, -1, 2)$ and is perpendicular to each of the planes $2x +3y -2z =5$ and $x +2y -3z =8$.
\item Find the distance between the point $P(6, 5, 9)$ and the plane determined by the points $A(3, -1 2), B( 5, 2,4)$ and $C(-1, -1, 6)$
\item Show that the lines 
\begin{align}
\frac{x-a+d}{\alpha -\delta} =\frac{y-a}{\alpha} =\frac{z-a-d}{\alpha +\delta}\\
 \text{  and } \frac{x-a+c}{\beta -\gamma} =\frac{y -b}{\beta} =\frac{z-b-c}{\beta +\gamma} \text{ are coplanar }.
\end{align}
\item Find the coordinates of the point where the line through the points $A(3,4,1)$ and $B(5, 1, 6)$ crosses XY-plane.
\end{enumerate}
\end{document}
