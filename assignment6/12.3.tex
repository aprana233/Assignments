\let\negmedspace\undefined
\let\negthickspace\undefined
\documentclass{article}
\usepackage{cite}
\usepackage{amsmath,amssymb,amsfonts,amsthm}
\usepackage{algorithmic}
\usepackage{graphicx}
\usepackage{textcomp}
\usepackage{xcolor}
\usepackage{txfonts}
\usepackage{listings}
\usepackage{enumitem}
\usepackage{tfrupee}
\usepackage{array}
\usepackage{tabularx}
\usepackage{mathtools}
\usepackage{gensymb}
\usepackage[breaklinks=true]{hyperref}
\usepackage{tkz-euclide} % loads  TikZ and tkz-base
\usepackage{listings}
\usepackage{gvv}
%
%\usepackage{setspace}
%\usepackage{gensymb}
%\doublespacing
%\singlespacing

%\usepackage{graphicx}
%\usepackage{amssymb}
%\usepackage{relsize}
%\usepackage[cmex10]{amsmath}
%\usepackage{amsthm}
%\interdisplaylinepenalty=2500
%\savesymbol{iint}
%\usepackage{txfonts}
%\restoresymbol{TXF}{iint}
%\usepackage{wasysym}
%\usepackage{amsthm}
%\usepackage{iithtlc}
%\usepackage{mathrsfs}
%\usepackage{txfonts}
%\usepackage{stfloats}
%\usepackage{bm}
%\usepackage{cite}
%\usepackage{cases}
%\usepackage{subfig}
%\usepackage{xtab}
%\usepackage{longtable}
%\usepackage{multirow}
%\usepackage{algorithm}
%\usepackage{algpseudocode}
%\usepackage{enumitem}
%\usepackage{mathtools}
%\usepackage{tikz}
%\usepackage{circuitikz}
%\usepackage{verbatim}
%\usepackage{tfrupee}
%\usepackage{stmaryrd}
%\usetkzobj{all}
%    \usepackage{color}                                   >
%    \usepackage{array}                                   >
%    \usepackage{longtable}                               >
%    \usepackage{calc}                                    >
%    \usepackage{multirow}                                >
%    \usepackage{hhline}                                  >
%    \usepackage{ifthen}                                  >
  %optionally (for landscape tables embedded in another do>
%    \usepackage{lscape}
%\usepackage{multicol}
%\usepackage{chngcntr}
%\usepackage{enumerate}

%\usepackage{wasysym}
%\documentclass[conference]{IEEEtran}
%\IEEEoverridecommandlockouts
% The preceding line is only needed to identify funding in>

\newtheorem{theorem}{Theorem}[section]
\newtheorem{problem}{Problem}
\newtheorem{proposition}{Proposition}[section]
\newtheorem{lemma}{Lemma}[section]
\newtheorem{corollary}[theorem]{Corollary}
\newtheorem{example}{Example}[section]
\newtheorem{definition}[problem]{Definition}
%\newtheorem{thm}{Theorem}[section]
%\newtheorem{defn}[thm]{Definition}
%\newtheorem{algorithm}{Algorithm}[section]
%\newtheorem{cor}{Corollary}
\newcommand{\BEQA}{\begin{eqnarray}}
\newcommand{\EEQA}{\end{eqnarray}}
%\newcommand{\define}{\stackrel{\triangle}{=}}
\theoremstyle{remark}
\newtheorem{rem}{Remark}

%\bibliographystyle{ieeetr}
\begin{document}
\title{Latex Assignment6}                            \author{APARNA ANAND}
\date{31 August,2023}
\maketitle
\section*{Example:-1-28 (12.3)}
\begin{enumerate}
\item Consider the following information regarding the number of men ad women workers in three factories I, II and III 
\begin{table}[ht!]
\centering
\begin{tabular}{|c|c|c|}
\hline
 & Men Workers & Women Workers\\
\hline
I &30&27\\
\hline
II  &25&31\\
\hline
III &27&26\\
\hline
\end{tabular}
\caption{}
\end{table}\\
Represent the above information in the form of a $3\times 2$ matrix. What does the entry in the third row and second column represent $?$
\item If a matrix has $8$ elements, what are the possible orders it can have ?
\item Construct a $3\times 2$ matrix whose elements are given by $a_{ij}= \frac{1}{2}\abs{1-3j}$
\item If $\myvec{x+3&z+4&2y-7\\-6&a-1&0\\b-3&-21&0}=\myvec{0&6&3y-2\\-6&-3&2c+2\\2b+4&-21&0}$. Find the values of $a,b,c,x,y$ and $z$. 
\item Find the values of $a, b, c$ and $d$ from the following equation :
\begin{align} 
\myvec{2a+b&a-2b\\5c-d&4c+d} = \myvec{4&-3\\11&24} 
\end{align}
\item Given $A=\myvec{\sqrt{3}&1&-1\\2&3&0}$ and $B=\myvec{2&\sqrt{5}&1\\-2&3&\frac{1}{2}}$, find $A+B$.
\item If $A=\myvec{1&2&3\\2&3&1}$ and $B=\myvec{3&-1&3\\-1&0&2}$, then find $2A-B$.
\item If $A=\myvec{8&0\\ 4&-2\\3&6}$ and $B=\myvec{2&-2\\4&2\\-5&1}$ then find that $X$, such that $2A+3X=5B$
\item Find $X$ and $Y$, if $X+Y=\myvec{5&2\\0&9}$ and $X-Y=\myvec{3&6\\0&-1}$
\item Find the values of $x$ and $y$ from the following equations:
\begin{align} 2 \myvec{x&5\\7&y-3}+\myvec{3&-4\\1&2}=\myvec{7&6\\15&14} \end{align}
\item Two farmers Ramkishan and Gurucharan Singh cultivate only three varities of rice namely Basmati, Permal and Naura. The sale (in Rupees) of these three varities of rice by both the farmers in the month of September and October are given by the following matrices $A$ and $B$.
 \\ September Sales (in Rupees)
\begin{align}
A=\myvec{ \text{Basmati} & \text{Permal} & \text{Naura}\\ 10000&20000&30000\\50000&30000&10000} \begin{array}{c}\text{Ramkishan}\\ \text{Gurucharan Singh}\end{array}
\end{align}
  October Sales (in Rupees)
\begin{align}
B=\myvec{\text{Basmati}& \text{Permal} &\text{Naura}\\5000&10000&6000\\20000&10000&10000}\begin{array}{c}\text{Ramakishan} \\ \text{Gurucharan Singh}\end{array}
\end{align}
\begin{enumerate}[label=(\roman*)]
\item Find the combined sales in Sepember and October for each farmer in each variety.
\item Find the decrease in sales from September to October.
\item If both farmers receive $2\%$  profit on gross sales, compute the profit for each farmer and for each variety sold in October.
\end{enumerate}
\item Find $AB$, if $A=\myvec{6&9\\2&3}$ and $B=\myvec{2&6&0\\7&9&8}$.
\item If $A=\myvec{1&-2&-3\\ -4&2&5}$ and $B=\myvec{2&3\\4&5\\2&1}$, then find $AB, BA$. Show that $AB\neq BA$.
\item If $A=\myvec{1&0\\0&-1}$ and $B=\myvec{0&1\\1&0}$, then $AB=\myvec{0&1\\-1&0}$ and $BA=\myvec{0&-1\\1&0}$ clearly $AB\neq BA$. Thus matrix multiplication is not commutative.
\item Find $AB$, if $A=\myvec{0&-1\\0&2}$ and $B=\myvec{3&5\\0&0}$.
\item If $A=\myvec{1&1&-1\\2&0&3\\3&-1&2}, B=\myvec{1&2\\0&2\\-1&4}$ and $C=\myvec{1&2&3&-4\\2&0&-2&1}$ find $A(BC), (AB)C$ and show that $(AB)C = A(BC)$.
\item If $A=\myvec{0&6&7\\-6&0&8\\7&-8&0}, B=\myvec{0&1&1\\1&0&2\\1&2&0}, C=\myvec{3\\-2\\3}$ Calculate $AC,BC$ and $(A+B)C$. Also,  verify that $(A+B)C=AC+BC$.
\item If $A=\myvec{1&2&3\\3&-2&1\\4&2&1}$, then show that $A^3-23A-40I=0$.
\item In a legislative assembly election, a political group hired a public relations firm to promote its candidate in three ways: telephone, housecalls and letters. The cost per contact (in paise) is given in matrix $A$ as \\ cost per contact  
\begin{align} A=\myvec{40\\100\\50}\begin{array}{c} \text{Telephone}\\ \text{Housecall}\\ \text{Letter}\end{array}\end{align}.
\\ The number of contacts of each type made in two cities $X$ and $Y$ is given by 
\begin{align} B=\myvec{\text{Telephone}&\text{Housecall}&\text{Letter} \\ 1000&500&5000\\3000&1000&10000}\begin {array}{c} X\\Y\end{array}\end{align}. 
\\ Find the total amount spent by the group in the two cities $X$ and $Y$.
\item If $A=\myvec{3&\sqrt3&2\\4&2&0}$ and $B=\myvec{2&-1&2\\1&2&4}$, verify that 
\begin{enumerate}
\item $(A^1)^1=A$
\item $(A+B)^1=A^1+B^1$
\item $(kB)^1=kB^1$, where $k$ is any constant.
\end{enumerate}
\item If $A=\myvec{-2\\4\\5}, B=\myvec{1&3&-6}$, verify that $(AB)^1=B^1A^1$.
\item Express the matrix $B=\myvec{2&-2&-4\\-1&3&4\\1&-2&-3}$ as the sum of symmetric and a skew symmetric matrix.
\item By using elementary operations, find the inverse of the matrix $A=\myvec{1&2\\2&1}$.
\item Obtain the inverse of the following matrix using elementary operations. 
\begin{align}A=\myvec{0&1&2\\1&2&3\\3&1&1} \end{align}.
\item Find $P^{-1}$, if it exists, given $P=\myvec{10&-2\\-5&1}$.
\item If $A=\myvec{\cos \theta& \sin \theta\\ -\sin \theta&\cos\theta}$, then prove that $A^n=\myvec{\cos n\theta&\sin n\theta\\ -\sin n\theta&\cos n\theta}, n\varepsilon N$.
\item If $A$ and $B$ are symmetric matrices of the same order, then show that $AB$ is symmetric if and only if $A$ and $B$ commute, that is $AB=BA$.
\item Let $A=\myvec{2&-1\\3&4}, B=\myvec{5&2\\7&4}, C=\myvec{2&5\\3&8}$. Find a matrix $D$ such that $CD-AB=0$.
\end{enumerate}
\end{document}
